
\section{Аналитическое решение}
\subsection{Линейный случай}
Решение системы, которую мы рассматриваем, имеет вид
\[
 u(t,x) = 
\begin{cases}
  0,& x+\frac t2\le 0;\\
  4x + 2t, &0<x+\frac t2 \le \frac14;\\
  1,& x+\frac t2>\frac14.
\end{cases}
\]
\subsection{Нелинейный случай}
Решение нелинейного уравнения $\CP ut - u\CP ux=0$ с начальным условием $u(0,x) = u_0(x)$ будем проводить в два этапа.
\begin{itemize}
\item Формально найдём решение $\phi(t,x)$ уравнения Хопфа $\CP ut + u\CP ux=0$ с начальным условием $\phi(0,x) = u_0(x)$. Решение задачи Коши в параметрическом виде
\[
  \begin{cases}
  \phi = u_0(\xi);\\
  x = \xi + u_0(\xi)t.
  \end{cases}
\]
Это решение в нашем случае имеет вид
\[
  \phi_1(t,x) = \begin{cases}
0,&x\le 0;\\
\frac{x}{0.25 + t},& 0<x<t+0.25;\\
1,& x\ge 0.25 +t.
\end{cases}
\]

В момент времени $t=-0.25$ происходит опрокидывание волны. Волна начинает двигаться вдвое медленнее по правилу Уизема. 
\[
  \phi_2(t,x) = \begin{cases}
0,x<\frac12(t+0.25);\\
1,x\ge\frac12(t+0.25).
\end{cases}
\]

\item Обратим время, то есть заметим, что искомое $u(t,x) = \phi(-t,x)$.
\[
  u(t,x) = \begin{cases}
\phi_1(-t,x),&t<\frac14;\\
\phi_2(-t,x),&t\ge \frac14.
\end{cases}
\]
\end{itemize}

Самое для нас интересное, это значение в момент времени $t=1$
\[
  u(1,x) = \begin{cases}
0,&x<-\frac38;\\
1,&x>\frac38.
\end{cases}
\]
А также выпишем условия на достаточно толстой границе
\[
  u(t,x) = \begin{cases}
    0,&x<-\frac38,\ t\in[0,1];\\
    1,&x>0.25,\ t\in[0,1];\\
\dotfill&\dotfill
\end{cases}
\]
